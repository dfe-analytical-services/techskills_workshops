% Options for packages loaded elsewhere
\PassOptionsToPackage{unicode}{hyperref}
\PassOptionsToPackage{hyphens}{url}
\PassOptionsToPackage{dvipsnames,svgnames,x11names}{xcolor}
%
\documentclass[
  12pt,
]{article}
\usepackage{amsmath,amssymb}
\usepackage{lmodern}
\usepackage{iftex}
\ifPDFTeX
  \usepackage[T1]{fontenc}
  \usepackage[utf8]{inputenc}
  \usepackage{textcomp} % provide euro and other symbols
\else % if luatex or xetex
  \usepackage{unicode-math}
  \defaultfontfeatures{Scale=MatchLowercase}
  \defaultfontfeatures[\rmfamily]{Ligatures=TeX,Scale=1}
\fi
% Use upquote if available, for straight quotes in verbatim environments
\IfFileExists{upquote.sty}{\usepackage{upquote}}{}
\IfFileExists{microtype.sty}{% use microtype if available
  \usepackage[]{microtype}
  \UseMicrotypeSet[protrusion]{basicmath} % disable protrusion for tt fonts
}{}
\makeatletter
\@ifundefined{KOMAClassName}{% if non-KOMA class
  \IfFileExists{parskip.sty}{%
    \usepackage{parskip}
  }{% else
    \setlength{\parindent}{0pt}
    \setlength{\parskip}{6pt plus 2pt minus 1pt}}
}{% if KOMA class
  \KOMAoptions{parskip=half}}
\makeatother
\usepackage{xcolor}
\usepackage[margin=1in]{geometry}
\usepackage{graphicx}
\makeatletter
\def\maxwidth{\ifdim\Gin@nat@width>\linewidth\linewidth\else\Gin@nat@width\fi}
\def\maxheight{\ifdim\Gin@nat@height>\textheight\textheight\else\Gin@nat@height\fi}
\makeatother
% Scale images if necessary, so that they will not overflow the page
% margins by default, and it is still possible to overwrite the defaults
% using explicit options in \includegraphics[width, height, ...]{}
\setkeys{Gin}{width=\maxwidth,height=\maxheight,keepaspectratio}
% Set default figure placement to htbp
\makeatletter
\def\fps@figure{htbp}
\makeatother
\setlength{\emergencystretch}{3em} % prevent overfull lines
\providecommand{\tightlist}{%
  \setlength{\itemsep}{0pt}\setlength{\parskip}{0pt}}
\setcounter{secnumdepth}{-\maxdimen} % remove section numbering
\usepackage{longtable}
\usepackage{graphics}
\usepackage{xparse}
\usepackage{moresize}
\usepackage{setspace}
\usepackage{tcolorbox}
\usepackage{wrapfig}
\usepackage{helvet}
\usepackage{sectsty}
\usepackage{fancyhdr}
\usepackage{xpatch}
\usepackage{booktabs}
\onehalfspacing
\pagestyle{fancy}
\definecolor{gssmidblue}{RGB}{32, 115, 188}
\definecolor{dfeheadingblue}{RGB}{16, 79, 117}
\renewcommand{\familydefault}{\sfdefault}
\allsectionsfont{\color{dfeheadingblue}}
\sectionfont{\color{dfeheadingblue}\fontsize{16}{18}\selectfont}
\fancyhead[C]{}
\fancyhead[RL]{}
\fancyfoot[LR]{}
\fancyfoot[C]{\sffamily \thepage}
\renewcommand{\headrulewidth}{0pt}
\renewcommand{\footrulewidth}{0pt}
\futurelet\TMPfootrule\def\footrule{{\color{gssmidblue}\TMPfootrule}}
\usepackage{floatrow}
\floatsetup[figure]{capposition=top}
\usepackage[tableposition=top]{caption}
\usepackage[titles]{tocloft}
\renewcommand{\cftdot}{}
\AtBeginDocument{\let\maketitle\relax}
\usepackage{cellspace}
\usepackage{etoolbox}
\colorlet{headercolour}{DarkSeaGreen}
\ifLuaTeX
  \usepackage{selnolig}  % disable illegal ligatures
\fi
\IfFileExists{bookmark.sty}{\usepackage{bookmark}}{\usepackage{hyperref}}
\IfFileExists{xurl.sty}{\usepackage{xurl}}{} % add URL line breaks if available
\urlstyle{same} % disable monospaced font for URLs
\hypersetup{
  pdftitle={DfE Statistics Development Team Workshops},
  pdfauthor={Department for Education},
  colorlinks=true,
  linkcolor={Maroon},
  filecolor={Maroon},
  citecolor={Blue},
  urlcolor={blue},
  pdfcreator={LaTeX via pandoc}}

\title{DfE Statistics Development Team Workshops}
\author{Department for Education}
\date{}

\begin{document}
\maketitle

\resizebox{48mm}{!}{\includegraphics{images/Department_for_Education.png}}

\vspace*{0.24\textheight}

\raggedright\HUGE{\color{dfeheadingblue}\textbf{DfE Statistics Development Team Workshops}} 

\huge{\color{dfeheadingblue}\textbf{Using git and GitHub (building a R-Shiny dashboard)}}
\vspace*{2\baselineskip} 

\normalsize 
 \newpage 


{
\hypersetup{linkcolor=}
\setcounter{tocdepth}{2}
\tableofcontents
}
\hypertarget{introduction}{%
\section{Introduction}\label{introduction}}

This document provides a workthrough guide for statistics publication
teams on how to work collaboratively using git, using the creation of a
data dashboard as a relevant context.

\hypertarget{background-information}{%
\section{Background Information}\label{background-information}}

\hypertarget{github-versus-dev-ops}{%
\subsection{GitHub versus Dev Ops}\label{github-versus-dev-ops}}

GitHub and Dev Ops effectively provide the same service in terms of
creating software via a git repository: they both act as the host for
the remote repository, whilst offering important tools to manage bugs
and issues, tasks, merging branches, deploying applications and so on.

\hypertarget{pre-workshop-requirements}{%
\section{Pre-workshop requirements}\label{pre-workshop-requirements}}

Here's a list of things you'll need to make sure are set up before
starting on this task sheet:

\begin{itemize}
\tightlist
\item
  GitHub account:
\item
  Install git on your laptop:
\item
  Install R-Studio on your machine
\end{itemize}

\hypertarget{setting-up-the-repository}{%
\section{Setting up the repository}\label{setting-up-the-repository}}

\hypertarget{creating-a-new-repository-on-github}{%
\subsection{Creating a new repository on
GitHub}\label{creating-a-new-repository-on-github}}

At this point, we're ready to create a new repository. The conext of
this exercise is to create a dashboard, so let's get a head start on
that by using the DfE R-Shiny template.

You can access the template here:

\url{https://github.com/dfe-analytical-services/shiny-template}

(images/gitdemo/gitdemo-shinydash-template.png){[}{]}

On that page, you'll see a button saying use this repo as a template. At
this one of your group (whilst logged in to GitHub) should click that
button, which will take you to the create repository page. Here you'll
have the option to create a copy of the template in your own GitHub
area.

\hypertarget{cloning-the-repository-to-your-local-machine}{%
\subsection{Cloning the repository to your local
machine}\label{cloning-the-repository-to-your-local-machine}}

\hypertarget{controlling-packages-with-renv}{%
\subsection{Controlling packages with
renv}\label{controlling-packages-with-renv}}

\hypertarget{basics-of-git}{%
\section{Basics of git}\label{basics-of-git}}

\hypertarget{adding-commiting-and-pushing}{%
\subsection{Adding, commiting and
pushing}\label{adding-commiting-and-pushing}}

\hypertarget{pulling-from-the-remote-repository}{%
\subsection{Pulling from the remote
repository}\label{pulling-from-the-remote-repository}}

\hypertarget{summary-of-git-basics}{%
\subsection{Summary of git basics}\label{summary-of-git-basics}}

\newpage

\resizebox{48mm}{!}{\includegraphics{images/Department_for_Education.png}}
\vspace*{\fill}
\color{black}

© Crown copyright 2022

This publication (not including logos) is licensed under the terms of
the Open Government Licence v3.0 except where otherwise stated. Where we
have identified any third party copyright information you will need to
obtain permission from the copyright holders concerned.

To view this licence:

\begin{tabular}{p{0.02\linewidth} p{0.1\linewidth} p{0.88\linewidth}}
& visit & www.nationalarchives.gov.uk/doc/open-government-licence/version/3 \\
& email & psi@nationalarchives.gsi.gov.uk \\
& write to & Information Policy Team, The National Archives, Kew, London, TW9 4DU \\
\end{tabular}

About this publication:

\begin{tabular}{p{0.02\linewidth} p{0.1\linewidth} p{0.88\linewidth}}
& enquiries & www.education.gov.uk/contactus \\
& download & www.gov.uk/government/publications \\
\end{tabular}

\begin{tabular}[t]{p{0.06\linewidth} p{0.24\linewidth} p{0.04\linewidth} p{0.06\linewidth} p{0.36\linewidth}}
\raisebox{-.5\height}{\includegraphics{images/logoTwitter.png}} &
Follow us on Twitter: @educationgovuk &
&
\raisebox{-.5\height}{\includegraphics{images/logoFacebook.png}} &
Like us on Facebook: \qquad facebook.com/educationgovuk\\
\end{tabular}

\end{document}
